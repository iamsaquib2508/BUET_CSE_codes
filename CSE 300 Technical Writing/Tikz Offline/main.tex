\documentclass{article}
\usepackage[utf8]{inputenc}
\usepackage[dvipsnames]{xcolor}
\usepackage{tikz}
\usetikzlibrary{arrows}
\usetikzlibrary{decorations.markings}
\usepackage{subfiles}
\usepackage{listings}
\usepackage{algorithm2e}
\definecolor{navyblue}{RGB}{14,12,169}
\definecolor{navyblue2}{RGB}{35,22,109}
\definecolor{lblue}{RGB}{115,113,234}
\definecolor{redd}{RGB}{247,49,4}

\title{Pythagorean Theorem}
\author{1505018}
\date{June 18, 2018}

\begin{document}

\maketitle
\newtheorem{theorem}{Theorem}[section]
\section{Introduction}

In this document, we present the very famous theorem in mathematics: \textit{Pythagorean theorem}, which is stated as follows.

\begin{theorem}[Pythagorean Theorem]
The square of the hypotenuse (the side opposite the right angle) is equal to the sum of the squares of the other two sides.
\end{theorem}
Numerous mathematicians proposed various proofs to the theorem. The theorem was long known even before the time of Pythagoras. Pythagoras was the first to provide with a sound proof. The proof that Pythagoras gave was by rearrangement. Even the great Albert Einstein also proved the theorem without rearrangement, rather by using dissection. Figure \ref{fig:pythagorean theorem} shows the visual representation of the theorem.

\subfile{fig2.tex}
\newpage
\subfile{fig3.tex}

\section{Trigonometric Forms}
Lots of other forms of the same theorem exist. The most useful, perhaps, are expressed in trigonometric terms, as follows:
\begin{equation}
    \sin^2{\theta} + \cos^2{\theta} = 1 
    \label{eqn:1}
\end{equation}
\begin{equation}
    \sec^2{\theta} - \tan^2{\theta} = 1 
    \label{eqn:2}
\end{equation}
\begin{equation}
    \mathrm{cosec}^2{\theta} - \cot^2{\theta} = 1 
    \label{eqn:3}
\end{equation}


\subsection{Representing the First}
Taking \ref{eqn:1}, we can show them as shown in Figure \ref{fig:pythagorean theorem2}. When we take a point at unit distance from the origin, the $y$ and $x$ co-ordinates become $\sin\theta$ and $\cos\theta$ respectively. Therefore, sum of the squares of the two becomes equal to the square of the unit distance, which of course, is $1$.

\end{document}
