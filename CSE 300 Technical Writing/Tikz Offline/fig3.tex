
\begin{figure}[h]
    \centering
    \begin{tikzpicture}[scale=0.7]
        
        \draw [navyblue2, line width=1.5pt] (0,0) circle [radius=5];
        
        \draw[lblue] (4,3) -- (0,3);
        \draw[lblue] (4,3) -- (4,0);
        \draw[redd, thick] (4,3) -- (0,0);
        \node [above, redd, rotate=36.8689, thick] at (2,1.5) {$r=1$};
        %\node[drop shadow,fill=mycolor,draw,rounded corners]
        
        \draw[navyblue2] (0,6) -- (0,-6);
        \draw[navyblue2] (-6,0) -- (6,0);
        
        \draw[navyblue] (4,3) -- (4,4);
        \draw[navyblue] (4,3) -- (6,3);
        
        
        \draw[navyblue, -triangle 90,fill=navyblue] (0,3.5) -- (4,3.5);
        \draw[navyblue, -triangle 90,fill=navyblue] (4,3.5) -- (0,3.5);
        \node [above, thick] at (2,3.5) {$\cos\theta$};
        
        \draw[navyblue, -triangle 90,fill=navyblue] (5.4,0) -- (5.4,3);
        \draw[navyblue, -triangle 90,fill=navyblue] (5.4,3) -- (5.4,0);
        \node [right, thick] at (5.4,1.5) {$\sin\theta$};
        \draw [lblue] (1,0) arc [radius = 1, start angle =0, end angle = 36.8689];
        \node [above right, lblue] at (1,0.1) {$\theta$};
        %\draw [ultra thick, help lines] (-10,-10) grid (10,10);
        
        
    \end{tikzpicture}
    \caption{Alternate representation of Pythagorean theorem.}
    \label{fig:pythagorean theorem2}
\end{figure}
